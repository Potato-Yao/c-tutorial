\section{流程控制 循环语句}

\subsection{while语句}

我们经常遇到这样的场景:需要一直做某个任务直到达到某个条件为止。在c语言中就有这样的设计,即循环语句。第一种循环是while语句,它的语法是while (条件) 下属语句。如果条件满足,就进入下属语句,下属语句执行完后程序会重新判断条件,如果还满足就再进入下属语句,直到条件不再满足为止。比如,我们若想要打印1到100,就可以这样做

\begin{lstlisting}[language=C]
    #include <stdio.h>

    int main() {
        int a = 1;
        while (a <= 100) {
            printf("%d\n", a);
        }

        return 0;
    }
\end{lstlisting}

我们首先定义了一个变量a为1,随后定义了一个循环语句,它打印a的值,直到a为100为止。这样,当a小于等于100时,下属语句就会一直执行,输出a的值,当a达到100后,循环退出,程序结束。看上去很完美不是?但是,假如我们运行这段代码(不建议大家尝试),你会发现电脑一直在输出1(有些电脑甚至会卡死)。

这是为什么?仔细想想,我们发现在循环的过程中,a的值始终是1,没有增长过,因此循环一直在执行而不会终止。所以,我们要在循环里加一条语句来使a增长。while语句改成了

\begin{lstlisting}[language=C]
    while (a <= 100) {
        printf("%d\n", a);
        ++a;
    }
\end{lstlisting}

再次运行,这次结果正确了。

我们说,流程控制语句是可以嵌套的,我们可以把if和while语句嵌套在一起,比如,可以这样输出1到100之间的全部偶数

\begin{lstlisting}[language=C]
    #include <stdio.h>

    int main() {
        int a = 1;
        while (a <= 100) {
            if (a % 2 == 0) {
                printf("%d\n", a);
            }
            ++a;
        }
        
        return 0;
    }
\end{lstlisting}

c语言提供了continue和break关键字,前者可以退出一次循环,后者用于直接退出整个循环。比如,上面的例子可以改写成

\begin{lstlisting}[language=C]
    #include <stdio.h>

    int main() {
        int a = 1;
        while (a <= 100) {
            if (a % 2 != 0) {
                ++a;
                continue;
            }
            printf("%d\n", a);
            ++a;
        }
        
        return 0;
    }
\end{lstlisting}

这是什么意思呢?每次在while下属语句时,都会首先判断a是不是奇数,如果是就进入if的下属语句,a自增,随后退出本次循环,因此下面的printf是不会被执行的。

再比如,假如我们有一个需求:输出一个数字,打印从1到这个数字之间的所有数字,如果打印到9就不再打印,那么就可以这样写

\begin{lstlisting}[language=C]
    #include <stdio.h>

    int main() {
        int a = 1;
        int b;
        printf("请输入一个大于1的整数\n");
        scanf("%d", &b);

        while (a <= b) {
            if (a == 9) {
                break;
            }
            
            printf("%d\n", a);
            ++a;
        }
        
        return 0;
    }
\end{lstlisting}

在循环的过程中,如果a增长到了9,就会进入if的下属语句,从而退出循环。这两个例子其实并没有体现出continue和break的真正作用,我们在之后的实战学习中会领略到这二者的真正作用。

值得注意的是,如果我们嵌套了多层循环语句,那么continue和break作用于最内层的循环。

有了循环的知识,我们就可以试着编写一个程序,它可以不间断地接受输入,直到输入了某个特定的退出符号为止。比如,我们可以编写一个程序来计算若干个正整数的平均数,如果输入非正数就退出循环。

为了实现这个程序,我们还是要遵循设计程序的三个步骤。首先确定需求:一,程序要能够不间断地输入,我们可以考虑使用while语句

\begin{lstlisting}[language=C]
    while (条件) {
        scanf("%d", &a);
    }
\end{lstlisting}

现在出现了两个问题:第一,while的条件应该是什么?第二,如果把输入结果存放在一个变量里,那么下一次的结果不就覆盖上一次的了?

第一个问题实际上就是我们的第二个需求,也即输入到非正数就退出,我们可以使用break语句,把程序写成这样

\begin{lstlisting}[language=C]
    while (条件) {
        scanf("%d", &a);
        if (a <= 0) {
            break;
        }
    }
\end{lstlisting}

如果这样,我们就不需要为循环设置条件了,因为我们的退出条件写在了循环内,所以代码就改成了

\begin{lstlisting}[language=C]
    while (1) {
        scanf("%d", &a);
        if (a <= 0) {
            break;
        }
    }
\end{lstlisting}

条件是1,意味着永远成立,因此这个循环如果内层没有退出条件的话,它会永远执行下去(也就是所谓的死循环)。第二个问题的解决其实很简单,反正我们最终的目的是要求和,我们可以这样写

\begin{lstlisting}[language=C]
    int sum = 0;
    int a;

    while (1) {
        scanf("%d", &a);
        if (a <= 0) {
            break;
        }
        sum += a;
    }
\end{lstlisting}

什么意思?我们每次输入的数字,都会被加到这个变量sum中,这样,我们最终得到的就是所有输入的数字的和了。为什么这行求和要写到if语句后,而不是写成一个else语句呢?也就是这样

\begin{lstlisting}[language=C]
    while (1) {
        scanf("%d", &a);
        if (a <= 0) {
            break;
        } else {
            sum += a;
        }
    }
\end{lstlisting}

这样当然也是可以的,它意味着:如果进入if的下属语句,就不会进入else的,反之如果进入else的下属语句就进入不了if的,这符合我们的需求。但是我们发现,在修改前的代码,假如进入if的下属语句就会退出循环,也执行不到求和的那一行,假如进入了求和的一行,就说明一定没有进入if,因此效果是一样的。

像上面这样,如果添加else语句与否的执行效果相同,我们建议不要使用else,这样就减少了嵌套的数量,增加了代码的可读性。因为,我们如果阅读的是这样一段代码

\begin{lstlisting}[language=C]
    if (a <= 0) {
        break;
    } else {
        sum += a;
    }
\end{lstlisting}

我们在阅读else下的代码时就要时刻惦记着:这是if语句的另一个情况,它代表了a是大于零的。我们每次都要想着if语句的条件才能对else的代码有正确的理解。但是如果写成这样

\begin{lstlisting}[language=C]
    if (a <= 0) {
        break;
    }
    sum += a;
\end{lstlisting}

我们就可以理解为:a小于等于零的情况在前面已经处理过了,我们断定下面的代码中a是大于零的,这段if语句就像是一个卫兵,向我们保证后面代码中a的取值范围,因此这样的,在核心业务逻辑代码前用来处理异常情况的代码被称做卫语句。这样就减少了阅读时的负担,这对于修改代码和理解代码都非常重要。实际上,我们一般推荐嵌套不要超过三层,我们在后期会讲到如何减少嵌套的数量。
