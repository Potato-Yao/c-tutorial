\section{流程控制 循环语句}

\subsection{while语句}

我们经常遇到这样的场景:需要一直做某个任务直到达到某个条件为止。在c语言中就有这样的设计,即循环语句。第一种循环是while语句,它的语法是while (条件) 下属语句。如果条件满足,就进入下属语句,下属语句执行完后程序会重新判断条件,如果还满足就再进入下属语句,直到条件不再满足为止。比如,我们若想要打印1到100,就可以这样做

\begin{lstlisting}[language=C]
    #include <stdio.h>

    int main() {
        int a = 1;
        while (a <= 100) {
            printf("%d\n", a);
        }

        return 0;
    }
\end{lstlisting}

我们首先定义了一个变量a为1,随后定义了一个循环语句,它打印a的值,直到a为100为止。这样,当a小于等于100时,下属语句就会一直执行,输出a的值,当a达到100后,循环退出,程序结束。看上去很完美不是?但是,假如我们运行这段代码(不建议大家尝试),你会发现电脑一直在输出1(有些电脑甚至会卡死)。

这是为什么?仔细想想,我们发现在循环的过程中,a的值始终是1,没有增长过,因此循环一直在执行而不会终止。所以,我们要在循环里加一条语句来使a增长。while语句改成了

\begin{lstlisting}[language=C]
    while (a <= 100) {
        printf("%d\n", a);
        ++a;
    }
\end{lstlisting}

再次运行,这次结果正确了。

我们说,流程控制语句是可以嵌套的,我们可以把if和while语句嵌套在一起,比如,可以这样输出1到100之间的全部偶数

\begin{lstlisting}[language=C]
    #include <stdio.h>

    int main() {
        int a = 1;
        while (a <= 100) {
            if (a % 2 == 0) {
                printf("%d\n", a);
            }
            ++a;
        }
        
        return 0;
    }
\end{lstlisting}

我们发现,我们为了循环而特意声明的变量a就像是一个计数器一样。实际上大部分时候这个变量的作用就是充当计数器,控制循环的进行。写计数器的时候一定不要忘记它的自增。这种充当计数器的变量一般命名为i、j和k。如果你在同一段代码内需要超过三个计数器,那么应该考虑是否要简化循环。

c语言提供了continue和break关键字,前者可以退出一次循环,后者用于直接退出整个循环。比如,上面的例子可以改写成

\begin{lstlisting}[language=C]
    #include <stdio.h>

    int main() {
        int a = 1;
        while (a <= 100) {
            if (a % 2 != 0) {
                ++a;
                continue;
            }
            printf("%d\n", a);
            ++a;
        }
        
        return 0;
    }
\end{lstlisting}

这是什么意思呢?每次在while下属语句时,都会首先判断a是不是奇数,如果是就进入if的下属语句,a自增,随后退出本次循环,因此下面的printf是不会被执行的。

再比如,假如我们有一个需求:输出一个数字,打印从1到这个数字之间的所有数字,如果打印到9就不再打印,那么就可以这样写

\begin{lstlisting}[language=C]
    #include <stdio.h>

    int main() {
        int a = 1;
        int b;
        printf("请输入一个大于1的整数\n");
        scanf("%d", &b);

        while (a <= b) {
            if (a == 9) {
                break;
            }
            
            printf("%d\n", a);
            ++a;
        }
        
        return 0;
    }
\end{lstlisting}

在循环的过程中,如果a增长到了9,就会进入if的下属语句,从而退出循环。这两个例子其实并没有体现出continue和break的真正作用,我们在之后的实战学习中会领略到这二者的真正作用。

值得注意的是,如果我们嵌套了多层循环语句,那么continue和break作用于最内层的循环。我们在学习到复杂嵌套时,各位可以自行验证这个结论。

有了循环的知识,我们就可以试着编写一个程序,它可以不间断地接受输入,直到输入了某个特定的退出符号为止。比如,我们可以编写一个程序来计算若干个正整数的平均数,如果输入非正数就退出循环。

为了实现这个程序,我们还是要遵循设计程序的三个步骤。首先确定需求:一,程序要能够不间断地输入,我们可以考虑使用while语句

\begin{lstlisting}[language=C]
    while (条件) {
        scanf("%d", &a);
    }
\end{lstlisting}

现在出现了两个问题:第一,while的条件应该是什么?第二,如果把输入结果存放在一个变量里,那么下一次的结果不就覆盖上一次的了?

第一个问题实际上就是我们的第二个需求,也即输入到非正数就退出,我们可以使用break语句,把程序写成这样

\begin{lstlisting}[language=C]
    while (条件) {
        scanf("%d", &a);
        if (a <= 0) {
            break;
        }
    }
\end{lstlisting}

如果这样,我们就不需要为循环设置条件了,因为我们的退出条件写在了循环内,所以代码就改成了

\begin{lstlisting}[language=C]
    while (1) {
        scanf("%d", &a);
        if (a <= 0) {
            break;
        }
    }
\end{lstlisting}

条件是1,意味着永远成立,因此这个循环如果内层没有退出条件的话,它会永远执行下去(也就是所谓的死循环)。第二个问题的解决其实很简单,反正我们最终的目的是要求和,我们可以这样写

\begin{lstlisting}[language=C]
    int sum = 0;
    int a;

    while (1) {
        scanf("%d", &a);
        if (a <= 0) {
            break;
        }
        sum += a;
    }
\end{lstlisting}

什么意思?我们每次输入的数字,都会被加到这个变量sum中,这样,我们最终得到的就是所有输入的数字的和了。为什么这行求和要写到if语句后,而不是写成一个else语句呢?也就是这样

\begin{lstlisting}[language=C]
    while (1) {
        scanf("%d", &a);
        if (a <= 0) {
            break;
        } else {
            sum += a;
        }
    }
\end{lstlisting}

这样当然也是可以的,它意味着:如果进入if的下属语句,就不会进入else的,反之如果进入else的下属语句就进入不了if的,这符合我们的需求。但是我们发现,在修改前的代码,假如进入if的下属语句就会退出循环,也执行不到求和的那一行,假如进入了求和的一行,就说明一定没有进入if,因此效果是一样的。

像上面这样,如果添加else语句与否的执行效果相同,我们建议不要使用else,这样就减少了嵌套的数量,增加了代码的可读性。因为,我们如果阅读的是这样一段代码

\begin{lstlisting}[language=C]
    if (a <= 0) {
        break;
    } else {
        sum += a;
    }
\end{lstlisting}

我们在阅读else下的代码时就要时刻惦记着:这是if语句的另一个情况,它代表了a是大于零的。我们每次都要想着if语句的条件才能对else的代码有正确的理解。但是如果写成这样

\begin{lstlisting}[language=C]
    if (a <= 0) {
        break;
    }
    sum += a;
\end{lstlisting}

我们就可以理解为:a小于等于零的情况在前面已经处理过了,我们断定下面的代码中a是大于零的,这段if语句就像是一个卫兵,向我们保证后面代码中a的取值范围,因此这样的,在核心业务逻辑代码前用来处理异常情况的代码被称做卫语句。这样就减少了阅读时的负担,这对于修改代码和理解代码都非常重要。实际上,我们一般推荐嵌套不要超过三层,我们在后期会讲到如何减少嵌套的数量。

如果不这么写呢?我们可以通过使用while的条件来控制循环的退出。即代码改成

\begin{lstlisting}[language=C]
    while (scanf("%d", &a), a > 0) {
        sum += a;
    }
\end{lstlisting}

这是什么意思?这段代码的核心是while括号内的那一行,这里涉及到了一个叫做逗号运算符的知识,即,在作为流程控制语句条件的语句里,我们可以用逗号连接几个语句,这几个语句依次执行,但是只有最后一个作为条件判断。就拿我们这段代码来说,我们使用逗号连接了scanf和a的大小判断。首先执行scanf,将输入的值赋给a,然后执行大小判断,因为这是最后一个语句,它的判断结果就作为while的条件判断了。逗号运算符是c语言一个很实用的知识,可以在不失去阅读性的情况下简化我们的代码。当然,我们一般不会在这里连接多个语句,一般就是像上面这样连接一个,否则代码过于臃肿。

现在我们有了所有数的总和,接下来要求平均值,这就需要我们知道具体有几个数,所以我们应该定义一个变量用来储存已经求了几个数的和。那么它应该写在哪里呢?我们记录的数的个数,是用来求和的数的个数,因此应该和求和写在一块,也就是

\begin{lstlisting}[language=C]
    int sum = 0;
    int count = 0;
    int a;

    while (1) {
        scanf("%d", &a);
        if (a <= 0) {
            break;
        }
        sum += a;
        ++count;
    }
\end{lstlisting}

或者使用逗号运算符的写法

\begin{lstlisting}[language=C]
    int sum = 0;
    int count = 0;
    int a;

    while (scanf("%d", &a), a > 0) {
        sum += a;
        ++count;
    }
\end{lstlisting}

最后就要计算并打印平均数了。别忘了,整数的平均值可以是小数,所以我们要定义一个浮点型来储存结果。另外,要先把sum转成浮点型再计算,这在之前就已经讲过原因了。因此,我们最终的程序是

\begin{lstlisting}[language=C]
    #include <stdio.h>

    int main() {
        int sum = 0;
        int a;
        int count = 0;

        while (scanf("%d", &a), a > 0) {
            sum += a;
            ++count;
        }

        double re = (double) sum / count;
        printf("%lf\n", re);
        
        return 0;
    }
\end{lstlisting}

当然了,如果是我们自己使用,那么可以增添一些提示性的文本,比如

\begin{lstlisting}[language=C]
    #include <stdio.h>

    int main() {
        int sum = 0;
        int a;
        int count = 0;

        printf("请输入连续若干个正整数,每输入一个就按一次回车。输入非正整数表示输入完成\n");

        while (scanf("%d", &a), a > 0) {
            sum += a;
            ++count;
        }

        double re = (double) sum / count;
        printf("以上数字的平均值是:%lf\n", re);
        
        return 0;
    }
\end{lstlisting}

其实c语言提供了一种类似于while的语句,叫做do while,比如

\begin{lstlisting}[language=C]
    do {
        printf("hi\n");
    } while (a > 5);
\end{lstlisting}

与while

\begin{lstlisting}[language=C]
    while (a > 5) {
        printf("hi\n");
    }
\end{lstlisting}

区别在于,do while要先执行一次do内的语句,然后再根据while判断是否循环。也就是说,do while会至少执行一次循环内的语句,而while不一定。

我们在使用while语句解决一些需要重复若干次循环的问题时,会发现它不太方便,比如我们想要打印10次Hello World

\begin{lstlisting}[language=C]
    int i = 1;
    while (i <= 10) {
        printf("Hello World\n");
        ++i;
    }
\end{lstlisting}

我们发现,为了循环计数,我们在外层定义了一个变量,还要在循环里时刻想着别忘了它的自增。这样非常麻烦,因此c语言提供了for语句,它的语法是 for (初始条件; 终止条件; 赋值语句) 下属语句。比如上面的例子,我们可以这样使用for语句实现

\begin{lstlisting}[language=C]
    for (int i = 0; i < 10; ++i) {
        printf("Hello World\n");
    }
\end{lstlisting}

这是什么意思呢?在执行for语句的时候,首先执行初始条件,创建了一个叫做i的变量,随后执行终止条件,发现满足,因此执行下属语句。下属语句执行完了,就会执行赋值语句,随后再执行终止条件。如果仍满足条件就继续执行下属语句,然后再执行赋值语句,如此一直循环,直到不满足终止条件为止,此时退出循环。

在我们的示例中,我们首先创建了变量i,然后到了终止条件,发现满足i小于10,因此执行下属语句,打印了一次,随后执行赋值语句,i变成了1,一直这样执行,直到i变成了10,此时不满足终止条件了,退出循环。我们从0一直执行到9,执行了10次。

那么,为什么不像while的例子那样,写成这样

\begin{lstlisting}[language=C]
    for (int i = 1; i <= 10; ++i) {
        printf("Hello World\n");
    }
\end{lstlisting}

从1开始,这两段代码效果相同,但是后者不是更好理解吗?其实不是,计算机世界往往从0开始计数(学习指针时会解释为什么),我们这样写,还节省了一个等号,这样代码看上去反而更简洁有效。毕竟,区间$[1, 10]$内的整数个数就等于$[0, 9]$的,因此也等于$[0, 10)$的。

就是说,如果我们想要重复n次,那么代码就写成

\begin{lstlisting}[language=C]
    for (int i = 0; i < n; ++i)
\end{lstlisting}

这样的话,i就相当于一个计数器一样。值得注意的是,定义的临时变量只在for循环内有效,即

\begin{lstlisting}[language=C]
    for (int i = 0; i < n; ++i) {
        // 这里可以使用i
    }
    // 这里不能用i
\end{lstlisting}

在for循环的下属语句内,我们也可以改变i的值。另外for也可以使用continue和break,比如我们在while写的输出1到100内偶数的例子,使用for可以改成

\begin{lstlisting}[language=C]
    #include <stdio.h>

    int main() {
        for (int i = 1; i <= 100; ++i) {
            if (i % 2 != 0) {
                continue;
            }
            printf("%d\n", i);
        }
        
        return 0;
    }
\end{lstlisting}

为什么i又赋成1了?我们要灵活变通,i在这里不只是计数器,它还要被输出,因此这样写更好。

有意思的是,for的初始条件、终止条件和赋值语句都可以不写,而是使用别的地方的代码替代。比如,如果我们使用的是定义在别的地方的变量作为循环的变量,那么就可以这样写

\begin{lstlisting}[language=C]
    int a = 0;
    for (; a < n; ++a) {}
\end{lstlisting}

也即,我们空下了初始条件部分。如果我们退出循坏是通过下属语句的break,那么for循环就可以写成

\begin{lstlisting}[language=C]
    for (int a = 0;; ++a) {
        if (a >= n) {
            break;
        }
    }
\end{lstlisting}

再比如,如果我们对a的赋值在下属语句,那么可以不写赋值语句

\begin{lstlisting}[language=C]
    for (int a = 0; a < n;) {
        ++a;
    }
\end{lstlisting}

总之,不需要哪部分就不写哪部分即可,但是不要忘记使用分号隔开各个部分。当然了,我们还可以同时不写多个部分,这都取决于具体情况。另外,有时我们会见到这样的代码

\begin{lstlisting}[language=C]
    for (;;) {}
\end{lstlisting}

这就是一个死循环,等价于

\begin{lstlisting}[language=C]
    while (1) {}
\end{lstlisting}

我们在引入for的时候就解释了它相对于while的优势,那么while的优势是什么?我们发现,for语句更像是为了确定循环次数的情况而设计的,在不确定循环次数的情况下,while更加简单。这二者都很重要,在实际开发时要经过思考后选用合适的循环语句。
