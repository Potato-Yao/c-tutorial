\section{基本数据类型和数值运算}

\subsection{基本数据类型}

我们在上一章介绍变量的时候就已经使用了代表整数的int,但是只有整数显然是远远不够用的,c语言还提供了不同取值范围的整数、小数、字符和字符串。这些统称为基本数据类型。

在计算机科学中,整数一般被称作整型。根据取值范围的不同,总共有四种整型:short(短整型),int(整型),long(长整型)和long long,可以这样声明整型变量

\begin{lstlisting}[language=C]
    short a = 1;
    int b = 2;
    long c = 3;
    long long d = 4;
\end{lstlisting}

short的取值范围是$[-32768, 32767]$\footnote{以笔者电脑(64位win11,MSVC STL)的limits.h作为标准},int的取值范围是$[-327682147483648, 327672147483647]$,long和long long的就更大了。实际上还有一类无符号整型,就是只有非负数的整型。在整型前加一个unsigned就是对应的无符号整型

\begin{lstlisting}[language=C]
    // 无符号类型
    unsigned short a = 1;
    unsigned int b = 2;
    unsigned long c = 3;
    unsigned long long d = 4;
\end{lstlisting}

short的范围是$[-32768, 32767]$,而unsigned short的范围是$[0, 65535]$,相当于把负数部分都转移给正数了。其它类型对应的无符号类型也是一样。

介绍了这么多,笔者建议大家:除非你有明确的设计要求,对计算机了解比较清晰,非常清楚自己在写什么,否则就一律使用int。哪怕是明确了需要处理的数据都是非负也不要用无符号,哪怕是确定short就够包含自己需要的数据也要用int。

为什么我们这么说?首先,为什么c语言设计了这么多种整型和无符号这些东西呢?我们知道,在c语言诞生的年代,内存是十分紧张的资源,所以设计多种占用不同内存大小的整型也是便于程序员精准控制内存。但是现代计算机内存大多都是8GB,16GB甚至更高,已经没有必要过于珍惜内存了。

如果我们使用推荐外的类型,一方面会增大开发难度,另一方面容易诞生意料以外的问题。为什么呢?首先请考虑一个问题,以short为例,它的最大值是32767,那假如我们给一个short变量赋值32768会怎么样?我们写一段代码试一下

\begin{lstlisting}[language=C]
    #include <stdio.h>
    #include <limits.h>

    int main() {
        short a = SHRT_MAX + 1;  // SHRT_MAX就是short的最大值,即32767
        printf("%hd\n", a);
        
        return 0;
    } 
\end{lstlisting}

运行结果是-32768,也就是说,如果我们赋的值超过了数据类型取值范围的最大值,它就会从最小值开始赋。一样的,若我们赋的小于范围最小值,就会从最大值开始赋。这就像是把取值范围这根数轴首尾连接在了一起一样。类似的结果也会出现在其它类型,这是c语言的特殊设计。

想象一下,假如你在设计一个学生管理软件,为了节省内存,使用short来记录学生的编号(因为你觉得学校小,不可能超过三万人),但是使用你开发的系统的学校今年突然人数暴增,现在学校内有了32768人,那么最后一个同学的编号就是-32768,这位同学得有多无奈?因此,一般不要使用比int小的类型的。

既然使用short有这样的危害,为什么不都使用long呢?首先,实际场景中很少存在数据大到连int都装不下的情况(谁家学校有三百多亿人啊),因此没必要选这么大的;另一方面,如果需要处理的数据都是普遍极大的,一般都是由开发者自行实现适用的记录长数字的类型,使用long不划算。所以我们也一般不考虑比int大的类型。

那么为什么无论使用场景都不用无符号呢?首先,无符号的最大值和最小值也是连接在一起的,这种情况更刺激:假如一个学校有65536名学生,我们使用unsigned short来编号,那么最后一位和第一位的编号是一样的,就没办法区分这二人了。其次,我们有时设计的一些复杂算法可能出错,若算法应只输出正值却输出了负值,那么说明算法一定出了问题,假如我们使用无符号类型,就无法这样判断了。

为什么不同整型有不同取值范围呢?这实际上与其占用的比特数有关。c语言提供了sizeof运算符,用于获取一个数据类型占用的比特数(比特数是一个整型)。比如,我们可以这样获取int所占用的比特数

\begin{lstlisting}[language=C]
    #include <stdio.h>

    int main() {
        int s = sizeof(int);
        printf("%zd\n", s);
        
        return 0;
    } 
\end{lstlisting}

输出结果是4。类似的我们可以知道其它整型的大小

\begin{lstlisting}[language=C]
    #include <stdio.h>

    int main() {
        printf("%zd\n", sizeof(short));
        printf("%zd\n", sizeof(int));
        printf("%zd\n", sizeof(long));
        printf("%zd\n", sizeof(long long));

        printf("%zd\n", sizeof(unsigned short));
        printf("%zd\n", sizeof(unsigned int));
        printf("%zd\n", sizeof(unsigned long));
        printf("%zd\n", sizeof(unsigned long long));
        
        return 0;
    }
\end{lstlisting}

我们发现,占用比特数越大,储存数字的范围越大。这是一个定性的结果,我们在更深层次的课程会详细解释储存数据的原理、取值范围如何计算。

小数在计算机科学一般称作浮点型,常用的是double(双精度浮点型),还有一种叫做float,这个用法与double一致,但是我们推荐使用double。我们可以这样定义一个浮点型变量

\begin{lstlisting}[language=C]
    double a = 1.23;  // 小数点是英文句号
\end{lstlisting}

如果给浮点型赋一个整数(比如赋1),我们习惯这样赋值

\begin{lstlisting}[language=C]
    double b = 1.0;  // 也即要加上.0
\end{lstlisting}

如何打印浮点数呢?打印double的占位符是\%lf,打印float的占位符是\%f,比如

\begin{lstlisting}[language=C]
    #include <stdio.h>

    int main() {
        double a = 3.14;
        printf("%lf\n", a);
        
        return 0;
    }
\end{lstlisting}

输出结果是3.140000。默认的\%lf为我们保留了六位小数,如果我们把a改成3.1415926,输出结果是3.141593,不仅保留了六位小数,还做了四舍五入。实际上,\%f也可以用在打印double上,但是使用scanf时就必须要用\%lf了。因此我们推荐在处理double时不要混用,一律使用\%lf。

如果小数太长,我们只需要显示前两位,我们就可以这样使用占位符

\begin{lstlisting}[language=C]
    printf("%.2lf\n", a);
\end{lstlisting}

它的意思是要保留小数点后两位数字,还是很形象的吧?假如我们这么写,并且把a的值改成3.14

\begin{lstlisting}[language=C]
    printf("%5.2lf\n", a);
\end{lstlisting}

输出就变成了“ 3.14”,注意多出一个空格。对于\%a.bf和\%a.blf,小数点之前的数字代表最少要输出多少位(注意小数点也算一位),之后的数字代表要输出多少位小数。如果要求最少输出位数比数字本身有的位数还多,那就会在最前面输出空格来补齐差距。

还记得字面量吗?数字42是字面量,更具体地说,它是整型字面量;数字3.14是浮点型字面量;“xyz”是字符串字面量(字符串在后续讲到)。总之,字面量可以更精确地描述为类型+字面量。

\subsection{数值运算}

整型和浮点型都具有加减乘除四种数值运算,整型还有取模运算。一个整型可以与另一个整型做数值运算,也可以与一个整型字面量数值运算,两个整型字面量也可以进行运算。加减乘除的符号分别是$+-*/$,就像数学的写法一样,在运算符两边写参与运算的数字。像是算一加一就是$1+1$,算二乘二就是$2*2$。另外需要特别说明,赋值号只在赋值号右边的运算都完成后才会把右边的值赋给左边。举个例子

\begin{lstlisting}[language=C]
    #include <stdio.h>

    int main() {
        int a = 1;  // a和b都是整型变量
        int b = 2;
        int c = a + b;  // 整型变量进行数值运算,得到的值赋给了c。注意:只有在赋值号右边的运算完成后,赋值号才进行赋值
        printf("%d\n", c);  // 输出结果是3

       	int d = 1;
        int e = d * 4;  // e由一个整型变量和整型字面量计算得到
        printf("%d\n", e);  // 输出结果是4

        int f = 2 - 2;  // f由两个整型字面量计算得到
        printf("%d\n", f);   // 输出结果是0
        
        return 0;
    }
\end{lstlisting}

最需要强调的是除法运算。整型的运算只能算出整型,所以假如参与除法运算的两个数不能整除怎么办?我们试一试

\begin{lstlisting}[language=C]
    #include <stdio.h>

    int main() {
        int a = 2;
        int b = 3;
        int c = a / b;
        printf("%d\n", c);
        
        return 0;
    }
\end{lstlisting}

照理来说,$2/3$的结果是0.666循环,但是这段代码的输出是0。这说明,整型的除法在无法整除的时候,会截取整数部分作为运算结果。另外考虑一个问题,以0作为除数会怎么样?0作为除数是没有意义的,在c语言中以0为除数的后果就会导致未定义行为。

除此以外,整型还有取模运算,也就是获取两数相除的余数。取模运算的符号是\%。比如

\begin{lstlisting}[language=C]
    #include <stdio.h>

    int main() {
        int a = 20;
        int b = 3;
        int c = a % b;
        printf("%d\n", c);
        
        return 0;
    }
\end{lstlisting}

输出结果是2,说明20/3的结果余2。这个运算非常重要,比如我们可以通过一个数与2的余数是否为0来判断这个数是否为偶数:如果余数为0,说明这个数能被2整除,那就是偶数,反之就是奇数。

浮点数的加减乘除运算语法和整型的是一致的,不过浮点数自然没有取模运算。我们刚刚尝试整型的除法运算,发现运算结果是截取了商的整数部分,那浮点型会怎样呢?

\begin{lstlisting}[language=C]
    #include <stdio.h>

    int main() {
        double a = 2.0;
        double b = 3.0;
        double c = a / b;
        printf("%lf\n", c);
        
        return 0;
    }
\end{lstlisting}

运行,发现结果是0.666657,说明浮点型的运算结果也是浮点型。实际上,整型与浮点型运算的结果也是浮点型,常见的是使用一个浮点型除以一个整型字面量,比如我们这样求三个数的平均值

\begin{lstlisting}[language=C]
    double a = 2.0;
    double b = 3.0;
    double c = 4.0;

    double avg = (a + b + c) / 3;
\end{lstlisting}

在c语言中,不同的数据类型可以进行转换。比如我们可以这样将一个整型转成浮点型

\begin{lstlisting}[language=C]
    int a = 1;
    double b = (double) a;  // b = 1.0,需要转成什么类型就在括号里写什么类型
\end{lstlisting}

这样,就将a转换成了一个浮点型变量并赋给了b。注意,转换的意义是生成了一个浮点型变量,而不是将a本身变成了浮点型。

一样的,我们可以把浮点型转换成整型。但是这样就有一个问题,比如我们将4.5转成整型,结果是多少呢?我们写一段代码试试

\begin{lstlisting}[language=C]
    #include <stdio.h>

    int main() {
        double a = 4.5;
        int b = (int) a;

        printf("%d\n", b);
        
        return 0;
    }
\end{lstlisting}

结果是4。由此我们发现,在浮点型转换为整型时,会直接截去小数点。

我们可以想到,浮点型能储存的数字比整型多,因此整型向浮点型转化是一定没有问题的,但是浮点型转整型可能会丢失小数点后的信息。也是由此,整形转浮点型是不需要像上面那样明确写出来的,编译器会自动帮我们进行转换,这叫做隐式转换。而浮点向整型的转换是需要我们按照上面的方法手动进行的,叫做显式转换。

c语言中算数是有优先级的,与数学一样,即从左到右,先算乘除后算加减,括号可以改变运算顺序。比如

\begin{lstlisting}[language=C]
    int a = 4 * 2 + 2;
	int b = 4 * (2 + 2);
\end{lstlisting}

a的值是10,b的值是16。注意,如果你需要套多层括号来改变运算顺序,都使用圆括号(即())而不要使用方括号和花括号。

我们刚刚提到了赋值号是先对右边进行运算再赋给左边,所以我们实际上可以写出这样的东西

\begin{lstlisting}[language=C]
    int a = 1;
    a = a + 1;
\end{lstlisting}

首先对右边计算,是1+1=2,随后将2赋给了a,此时a就变成2了。当然了,对于其它几种数值运算也能这样使用。

像上面这样的运算是很常用的,但是每次都写成那样岂不是很麻烦?c语言提供了一种简化语法

\begin{lstlisting}[language=C]
    int a = 1;
    a += 1;  // 等价于 a = a + 1;
    // 类似的,a *= 2就等价 a = a * 2,诸如此类
\end{lstlisting}

在后续学习到流程控制时,我们需要常常对一个整型变量进行加一或者减一的运算,此时哪怕是上面的简化语法也太累赘了,因此c语言设计了自增运算符和自减运算符,用于对整型进行加一或减一的操作(其实对浮点型也适用,不过最常用于整型)。

自增运算符和自减运算符都分为前置和后置两种

\begin{lstlisting}[language=C]
    int a = 1;
    ++a;  // 前置自增运算符
    a++;  // 后置自增运算符
    --a;  // 前置自减运算符
    a--;  // 后置自减运算符
\end{lstlisting}

无论是哪种自增运算符,都能使变量的值加一。无论是哪种自减,都能使值减一。

前置和后置运算符的区别在于将它们与赋值并用的时候

\begin{lstlisting}[language=C]
    int a = 1;
    // 下面两段代码不能放在一起,这里仅作为演示
    int j = ++a;  // j = 2
    int j = a++;  // j = 1
\end{lstlisting}

当赋值运算和前置运算符并用时,会先自增后赋值,因此j的值就是2。当与后置运算符并用时,会先赋值后自增,所以j获得的是原来的值1。

记不住怎么办?记不住就老老实实写成这样

\begin{lstlisting}[language=C]
    int a = 1;
    ++a;
    int j = a;
\end{lstlisting}

另外,虽然使用上面这样的写法就无需考虑使用前置还有后置运算符了,但是笔者推荐养成使用前置运算符的习惯,这样程序的效率更高。这是从C++中获取的经验,不多过解释了。(虽然对于Java等语言来说这两种运算符是完全等价的,但是对于C和C++来说还是有区别)

最后,我们写一个实例:让用户输入三个整数,计算它们的平均值并返回。

对于这种实战型的题目,我们的一般方法就是先明确设计需求,然后大概确定技术要点,最后进行开发即可。

对于我们这个实例,设计需求就是:一,获取三个整数;二,计算三个数字的平均数并输出。那么这涉及什么技术呢?对于第一个需求,需要掌握scanf的用法,对于第二个,需要掌握变量的数值运算和printf的用法。我们都学过,所以开始编程吧。

首先,我们创建新的main.c,输入必要的代码

\begin{lstlisting}[language=C]
    #include <stdio.h>

    int main() {
        return 0;
    }
\end{lstlisting}

我们首先处理第一个需求。获取三个整数,这就需要我们把它们储存起来,因此先定义三个整型变量

\begin{lstlisting}[language=C]
    int a;
    int b;
    int c;
\end{lstlisting}

接下来就是要获取三个数字并进行储存了。我们可以写成这样

\begin{lstlisting}[language=C]
    scanf("%d %d %d", &a, &b, &c);
\end{lstlisting}

这样,用户只需要按照模板进行输入,程序就可以正常工作。可是,用户该怎么知道这个程序是干什么的,他应该怎么输入呢?因此,我们在scanf前加一个printf来打印基本信息。这样,程序就改成了

\begin{lstlisting}[language=C]
    printf("欢迎使用平均数计算器!请按照a b c的格式输入三个整数\n");
    scanf("%d %d %d", &a, &b, &c);
\end{lstlisting}

在开发的过程中,应该开发一部分就进行一次测试。因此我们此时可以点击运行试试。发现出现了我们想要的界面,按照模板进行输入,程序以退出代码0退出。退出代码0是程序正常完成的标志。由此完成了第一个需求。

接下来我们需要计算平均值了,定义一个变量avg作为平均值,可是它应该是什么类型呢?我们想,三个整数的平均数不一定是一个整数,所以要用浮点型。因此就定义了

\begin{lstlisting}[language=C]
    double avg;
\end{lstlisting}

接下来就要计算平均数了,于是我们写下这样的代码

\begin{lstlisting}[language=C]
    double avg = (a + b + c) / 3;
\end{lstlisting}

然后输出

\begin{lstlisting}[language=C]
    printf("平均数为:%lf\n", avg); 
\end{lstlisting}

大功告成!我们运行程序,试一下效果。首先我们输入1 2 3,发现输出是2.000000,是我们预期的结果!那我们再试试1 2 4,发现结果不对了,怎么还是2.000000?

这说明,我们的代码出了错。回顾一下,输入部分的代码已经做了测试,输出数字也不会有问题,看来就是计算平均数这步代码出错了。那么为什么呢?注意,abc三个变量实际都是整型,它们的和是一个整型,除以3是整型的数值运算,得到的结果自然也是整型,也就是2。这个结果隐式转化成了浮点型,因此输出结果是2.000000。也就是说,我们的问题在于是使用整型进行运算的。如果我们把代码改成这样

\begin{lstlisting}[language=C]
    double avg = (double)(a + b + c) / 3;
\end{lstlisting}

这样,三个变量求和的结果会强制转换成浮点型,使用浮点型的运算算出的才是正确结果。修改后我们再次运行代码,发现这次的输出2.333333是正确的。这样我们就完成了整个项目的设计,完整代码如下

\begin{lstlisting}[language=C]
    #include <stdio.h>

    int main() {
        int a;
        int b;
        int c;

        printf("欢迎使用平均数计算器!请按照a b c的格式输入三个整数\n");
        scanf("%d %d %d", &a, &b, &c);

        double avg = (double)(a + b + c) / 3;

        printf("平均数为:%lf\n", avg);
        
        return 0;
    }
\end{lstlisting}
