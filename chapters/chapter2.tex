\section{基本数据类型和数值运算}

\subsection{基本数据类型}

我们在上一章介绍变量的时候就已经使用了代表整数的int,但是只有整数显然是远远不够用的,c语言还提供了不同取值范围的整数、小数、字符和字符串。这些统称为基本数据类型。

在计算机科学中,整数一般被称作整型。根据取值范围的不同,总共有四种整型:short(短整型),int(整型),long(长整型)和long long,可以这样声明整型变量

\begin{lstlisting}[language=C]
    short a = 1;
    int b = 2;
    long c = 3;
    long long d = 4;
\end{lstlisting}

short的取值范围是$[-32768, 32767]$\footnote{以笔者电脑(64位win11,MSVC STL)的limits.h作为标准},int的取值范围是$[-327682147483648, 327672147483647]$,long和long long的就更大了。实际上还有一类无符号整型,就是只有非负数的整型。在整型前加一个unsigned就是对应的无符号整型

\begin{lstlisting}[language=C]
    // 无符号类型
    unsigned short a = 1;
    unsigned int b = 2;
    unsigned long c = 3;
    unsigned long long d = 4;
\end{lstlisting}

short的范围是$[-32768, 32767]$,而unsigned short的范围是$[0, 65535]$,相当于把负数部分都转移给正数了。其它类型对应的无符号类型也是一样。

介绍了这么多,笔者建议大家:除非你有明确的设计要求,对计算机了解比较清晰,非常清楚自己在写什么,否则就一律使用int。哪怕是明确了需要处理的数据都是非负也不要用无符号,哪怕是确定short就够包含自己需要的数据也要用int。

为什么我们这么说?首先,为什么c语言设计了这么多种整型和无符号这些东西呢?我们知道,在c语言诞生的年代,内存是十分紧张的资源,所以设计多种占用不同内存大小的整型也是便于程序员精准控制内存。但是现代计算机内存大多都是8GB,16GB甚至更高,已经没有必要过于珍惜内存了。

如果我们使用推荐外的类型,一方面会增大开发难度,另一方面容易诞生意料以外的问题。为什么呢?首先请考虑一个问题,以short为例,它的最大值是32767,那假如我们给一个short变量赋值32768会怎么样?我们写一段代码试一下

\begin{lstlisting}[language=C]
    #include <stdio.h>
    #include <limits.h>

    int main() {
        short a = SHRT_MAX + 1;  // SHRT_MAX就是short的最大值,即32767
        printf("%hd\n", a);
        
        return 0;
    } 
\end{lstlisting}

运行结果是-32768,也就是说,如果我们赋的值超过了数据类型取值范围的最大值,它就会从最小值开始赋。一样的,若我们赋的小于范围最小值,就会从最大值开始赋。这就像是把取值范围这根数轴首尾连接在了一起一样。类似的结果也会出现在其它类型,这是c语言的特殊设计。

想象一下,假如你在设计一个学生管理软件,为了节省内存,使用short来记录学生的编号(因为你觉得学校小,不可能超过三万人),但是使用你开发的系统的学校今年突然人数暴增,现在学校内有了32768人,那么最后一个同学的编号就是-32768,这位同学得有多无奈?因此,一般不要使用比int小的类型的。

既然使用short有这样的危害,为什么不都使用long呢?首先,实际场景中很少存在数据大到连int都装不下的情况(谁家学校有三百多亿人啊),因此没必要选这么大的;另一方面,如果需要处理的数据都是普遍极大的,一般都是由开发者自行实现适用的记录长数字的类型,使用long不划算。所以我们也一般不考虑比int大的类型。

那么为什么无论使用场景都不用无符号呢?首先,无符号的最大值和最小值也是连接在一起的,这种情况更刺激:假如一个学校有65536名学生,我们使用unsigned short来编号,那么最后一位和第一位的编号是一样的,就没办法区分这二人了。其次,我们有时设计的一些复杂算法可能出错,若算法应只输出正值却输出了负值,那么说明算法一定出了问题,假如我们使用无符号类型,就无法这样判断了。

小数在计算机科学一般称作浮点型,常用的是double(双精度浮点型)。我们可以这样定义一个浮点变量

\begin{lstlisting}[language=C]
    double a = 1.23;  // 小数点是英文句号
\end{lstlisting}


