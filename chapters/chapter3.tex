\section{流程控制 选择语句}

\subsection{布尔代数基础}

我们一切命题,最终都会回到“是”与“否”的基本判断上。我们若使用1代表真,使用0代表假,便很容易得到一些逻辑运算。

首先是和关系。我们说,“我和那位同学都是女生”,这句话成立的条件是我是女生,并且那位同学也是女生。也就是说,和关系意味着两个条件都成立。

\begin{align*}
    0\quad and\quad 0 \Rightarrow 0\\
    1\quad and\quad 0 \Rightarrow 0\\
    0\quad and\quad 1 \Rightarrow 0\\
    1\quad and\quad 1 \Rightarrow 1
\end{align*}

类似地,或关系成立的条件是两个条件成立一个即可。比如“A或B是女生”,只要A与B之一是女生就成立,若都是女生也成立。

\begin{align*}
    0\quad or\quad 0 \Rightarrow 0\\
    1\quad or\quad 0 \Rightarrow 1\\
    0\quad or\quad 1 \Rightarrow 1\\
    1\quad or\quad 1 \Rightarrow 1
\end{align*}

还有一种基本的就是非关系。“我不是女生”若成立,就说明“我是女生”不成立,因此

\begin{align*}
    not\quad 0 \Rightarrow 1\\
    not\quad 1 \Rightarrow 0 
\end{align*}

使用以上三种运算,我们可以结合出很多复杂的运算。比如,“A是女生或B不是女生”,就是A是女生或者B是男生时成立。总之,这样的逻辑运算一定要能够想明白。

\subsection{if语句}

if语句用于条件判断,若判断为真,就进入下属的语句,反之就不进入。它的语法是if (表达式) 下属语句。这里的表达式将会是一个数字,如果这个数字为0就不进入,反之都进入。比如,我们可以这样写

\begin{lstlisting}[language=C]
    #include <stdio.h>

    int main() {
        int a = 1;

        if (a) printf("a的值不是0");
        printf("已经完成判断\n");

        return 0;
    }
\end{lstlisting}

输出结果是“a的值不是0已经完成判断”。因为用于判断的表达式的结果是数字1,可以进入下属语句。如果将a的值改为0,就不会进入下属语句了。

如果需要在下属语句部分写多行代码怎么办,这就需要我们使用花括号把代码括起来,就像是

\begin{lstlisting}[language=C]
    #include <stdio.h>

    int main() {
        int a = 1;

        if (a) {
            printf("a的值不是0\n");
            printf("a的值是%d\n", a);
        }
        printf("已经完成判断\n");

        return 0;
    }
\end{lstlisting}

实际上,我们推荐无论下属语句是一行还是多行,都是用花括号括起来。这样可以避免不必要的错误。

在c语言中,我们有一些用于对数字进行大小比较的运算符,它们是相等判断,不等判断和大于、小于、大于等于以及小于等于符号。我们常常在流程控制的语句中使用它们。

\begin{longtable}{c|c}
     关系 & 操作符\\
     \hline
     等于 & ==\\
     \hline
     不等 & !=\\
     \hline
     小于 & <\\
     \hline
     大于 & >\\
     \hline
     小于等于 & <=\\
     \hline
     大于等于 & >=\\
\end{longtable}

这些符号可以这样使用

\begin{lstlisting}[language=C]
    #include <stdio.h>

    int main() {
        int a = 5;
        int b;  // 储存输入的数字

        printf("输入一个数,比较其与5的大小关系!\n");
        scanf("%d", &b);

        // 判断a和b(也即预设的数字和输入的数字)大小关系
        if (a <= b) {
            printf("输入的数大于等于5\n");
        }
        if (a > b) {
            printf("输入的数小于5\n");
        }

        return 0;
    }
\end{lstlisting}

\begin{lstlisting}[language=C]
    #include <stdio.h>

    int main() {
        int a = 5;
        int b;  // 储存输入的数字
        printf("请输入一个数,猜猜和预设的是否一样\n");
        scanf("%d", &b);

        // 判断a和b(也即预设的数字和输入的数字)是否相等,如果相等就进入下属语句
        if (a == b) {
            printf("你猜对了!");
        }
        // 判断a和b(也即预设的数字和输入的数字)是否相等,如果不等就进入下属语句
        if (a != b) {
            printf("你猜错了!");
        }
        
        return 0;
    }
\end{lstlisting}

在讲取模运算符的时候,我们提到,它可以用来判断一个数字的奇偶性,我们来试试设计一个程序,输入一个整数,判断其奇偶性。

开发项目一定要按照我们之前提过的步骤进行。首先我们确定设计需求:一,输入一个整数;二,判断这个整数的奇偶性;三,根据奇偶性分别输出。第一个需求使用scanf即可,第二个使用我们学习过的取模运算符进行判断即可,第三个使用if语句就能实现。我们来编写代码。

第一个需求是比较简单的,我们直接给出代码

\begin{lstlisting}[language=C]
    #include <stdio.h>

    int main() {
        int a;
        printf("请输入一个整数\n");
        scanf("%d", &a);
        
        return 0;
    }
\end{lstlisting}

接下来就要判断奇偶性了。我们之前提到,一个数字若是偶数,那么它与2的模一定为0(因为偶数一定被2整除),因此对于输入的数字,如果与2取模的结果是0,那么就是偶数,反之就是奇数。因此我们首先写出了判断偶数的程序

\begin{lstlisting}[language=C]
    if (a % 2 == 0) {
        printf("你输入了一个偶数\n");
    }
\end{lstlisting}

那么判断奇数怎么写?仔细想想,一个数与2的余数,除了0,只能是1。因此

\begin{lstlisting}[language=C]
    if (a % 2 == 1) {
        printf("你输入了一个奇数\n");
    }
\end{lstlisting}

这样我们就完成了整个程序的设计。完整代码如下

\begin{lstlisting}[language=C]
    #include <stdio.h>

    int main() {
        int a;
        printf("请输入一个整数\n");
        scanf("%d", &a);

        if (a % 2 == 0) {
            printf("你输入了一个偶数\n");
        }
        if (a % 2 == 1) {
            printf("你输入了一个奇数\n");
        }
        
        return 0;
    }
\end{lstlisting}

上面的代码确实完成了我们的需求,但是我们发现,一个数要么是奇数,要么是偶数,我们其实只需要判断一次就能知道数的奇偶性,而不是需要像现在这样费力地构造逻辑上相反的语句。所以如果if语句可以有两个从属语句,如果条件满足就转向第一个,反之转向第二个,我们的程序就能大幅简化。c语言提供了这样的语法,就是else语句。else语句跟在if语句后,就像是

\begin{lstlisting}[language=C]
    if (a % 2 == 0) {
        printf("你输入了一个偶数\n");
    } else {
        printf("你输入了一个奇数\n");
    }
\end{lstlisting}

使用else语句改造后的代码就直观简洁多了。

我们之前提到了布尔代数,三种基本的布尔运算在c语言中也有对应的语法。

\begin{longtable}{c|c}
    布尔运算 & 符号\\
    \hline
    和运算 & \&\&\\
    \hline
    或运算 & ||\\
    \hline
    非运算 & !\\ 
\end{longtable}

比如说,我们可以这样判断一个数是否包含于$[10, 100]$的区间内。

\begin{lstlisting}[language=C]
    if (a >= 10 && a <= 100)
    // 注意,不能写成这样:if (10 <= a <= 100)
\end{lstlisting}

我们可以这样判断一个数是否同时满足不是3的倍数或是偶数

\begin{lstlisting}[language=C]
    if (a % 3 != 0 || a % 2 == 0)
\end{lstlisting}

我们之前写了这样的代码来判断一个数是否为偶数

\begin{lstlisting}[language=C]
    if (a % 2 == 0)
\end{lstlisting}

实际上,上面的式子可以改成这样

\begin{lstlisting}[language=C]
    if (!(a % 2))
\end{lstlisting}

这是在干什么?最内层括号计算了a与2的余数,如果a是偶数,那么得到的结果就是0,我们想要进入下属语句,因此把运算结果使用非运算反过来就可以了。一样的,如果我们输入的是奇数,得到的结果是1,非运算后就变成了0,就不会进入下属语句了。因此,我们将判断奇偶数改成了一种更加简洁的语法。注意不要遗漏了取模运算的括号,否则非运算先与a结合,就得不到我们期望的结果了。

有了布尔代数的知识,我们可以解决一类更复杂的问题。比如,给定一个成绩,要判断它属于不及格($[0, 60)$),及格($[60, 80)$),良好($[80, 90)$)还是优秀($[90, 100]$),我们怎么写呢?最开始的想法是这样的

\begin{lstlisting}[language=C]
    #include <stdio.h>

    int main() {
        int a;
        printf("请输入成绩\n");
        scanf("%d", &a);

        if (a >= 0 && a < 60) {
            printf("未及格!\n");
        }
        if (a >= 60 && a < 80) {
            printf("及格!\n");
        }
        if (a >= 80 && a < 90) {
            printf("良好!\n");
        }
        if (a >= 90 && a <= 100) {
            printf("优秀!\n");
        }
        
        return 0;
    }
\end{lstlisting}

这样有三个问题:一,假如输入的数字大于100,或者小于0怎么办?我们还需要写一个if来捕获这种情况;二,我们的程序是按照顺序执行的,比方说我们输入的数字是65,在第二个if已经捕获了,照理来说就不需要再去判断下面的几个if了,但是它还是会执行下去,这样浪费了很多性能;三,这一系列的if逻辑上是连续的,但是我们的代码表现出来是孤立的,这样增大了阅读难度。

怎么解决呢?c语言提供了else if语句,我们将上面的代码改成

\begin{lstlisting}[language=C]
    if (a >= 0 && a < 60) {
        printf("未及格!\n");
    }
    else if (a >= 60 && a < 80) {
        printf("及格!\n");
    }
    else if (a >= 80 && a < 90) {
        printf("良好!\n");
    }
    else if (a >= 90 && a <= 100) {
        printf("优秀!\n");
    }
    else {
        printf("输入数据有误\n");
    }
\end{lstlisting}

这样解决了我们提到的三个问题。else if按照从上到下的顺序执行,如果第一个if没有匹配到,就转到下一个if,直到匹配到为止。匹配到之后就执行匹配到的if的下属语句,而不会继续执行if语句。此外,我们还可以在最后缀一个else语句来处理缺省情况。

c语言提供了一个特殊的运算符,叫做三目运算符,用来快速给变量赋值。比如

\begin{lstlisting}[language=C]
    int a = b > 0 ? 4 : 5;
\end{lstlisting}

此处的英文问号和后面的英文冒号就是三目运算符了。它的含义是,当问号前的语句成立时,就使用冒号前的值,反之使用冒号后的。在我们的例子中,如果b>0是成立的,就使用冒号前的值,a的值就是4,如果b>0是不成立的,那么a就取5。这个运算符适用于上面例子中的这种简单场景,不要滥用,否则会增大代码的阅读难度。

流程控制语句是可以嵌套的,比如

\begin{lstlisting}[language=C]
    if (a > 0) {
        if (a % 2 == 0) {
            printf("a是一个大于0的偶数\n");
        } else {
            printf("a是一个大于0的奇数\n");
        }
    }
\end{lstlisting}

在嵌套时,else语句总是和与其最近的if语句匹配,这点我们观察大括号的嵌套也能看出来。我们在之后的学习和开发实战中会经常用到流程控制语句的嵌套。

最后,c语言还提供了一种switch语句,类似于if else,因为这个东西基本没用,我们就不过多介绍了。
