\section{流程控制 复杂实例}

\subsection{变量作用域}

考虑一段这样的程序

\begin{lstlisting}[language=C]
    #include <stdio.h>

    int main() {
        int a = 1;
        if (a) {
            int b = 1;
            if (b) {
                int c = 2;
                printf("%d\n", c);
            } else {
                int c = 3;
                printf("%d\n", c);
            }
        }
        
        return 0;
    }
\end{lstlisting}

运行结果是2,如果我们把b的值改成0,那么运行结果就是3。我们之前说过,变量不可以重复定义,可是这里命名重复定义了变量c,为什么这段代码可以正常执行呢?

我们把上面的代码抽象一下,提取出核心

\begin{lstlisting}[language=C]
    {
        int c = 2;
        printf("%d\n", c);
    }
    {
        int c = 3;
        printf("%d\n", c);
    }
\end{lstlisting}

我们发现,实际上两个c定义在了两个并列的花括号内。在c语言中,变量会在定义变量一级的花括号结束时被销毁。比如上面的代码,第一个变量c定义在第一个花括号内,因此当第一个花括号结束时,它就被销毁了,因此在第二个花括号中再定义一个c是不冲突的。

再考虑这种情况

\begin{lstlisting}[language=c]
    {
        int a;
        {
            int a;
        }
    }
\end{lstlisting}

这样就不行了。因为定义变量a的一级的花括号在代码的最后才结束,当我们第二次定义a的时候,之前的a还没有被销毁,所以这段代码是有错误的。

由于在第二级括号的时候a还没有被销毁,因此我们还是可以使用它的

\begin{lstlisting}[language=c]
    {
        int a = 1;
        {
            printf("%d\n", a);
        }
    }
\end{lstlisting}

但是内层的变量在外层就不能用了,比如

\begin{lstlisting}[language=c]
    {
        {
            int a = 1;
        }
        printf("%d\n", a);
    }
\end{lstlisting}

这是不行的,因为a在内层花括号结束时就被销毁了。

\subsection{跳转语句}

c语言提供了一种叫做goto的语句,用来进行跳转。要想使用goto,我们需要先定义一个标签,比如

\begin{lstlisting}[language=c]
    #include <stdio.h>

    int main() {
        int n = 1;
    label:
        printf("%d\n", n);
        ++n;

        return 0;
    }
\end{lstlisting}

然后使用goto label就可以跳转到标签的位置了,比如将上面的代码扩充为

\begin{lstlisting}[language=c]
    #include <stdio.h>

    int main() {
        int n = 1;
    label:
        printf("%d\n", n);
        ++n;
        if (n < 10) {
            goto label;
        }

        return 0;
    }
\end{lstlisting}

这段代码会输出1到9。这是因为当n小于10,进入if语句的时候,会执行下属的goto语句,跳转到label所在的地方,又重新开始向下执行代码。

笔者强烈建议永远不要使用goto语句,因为它会导致代码的可读性严重降低,产生难以捕获的bug。并且所有的label都只能在同一个函数中跳转,局限性大(非local的跳转需要使用setjump.h,非常复杂)。goto语句完全可以被使用恰当的break替代,因此不要使用goto。

\subsection{循环语句的复杂嵌套}

考虑这个需求:需要输入行数和列数,打印对应行数和列数的一个由1组成的矩形,比如输入2 3,打印

\begin{mdframed}
\begin{verbatim}
    11
    11
    11
\end{verbatim}
\end{mdframed}

怎么实现?我们可以考虑使用一个循环来控制行数,在每一行中用另一个循环来打印对应个数的1和换行符。因此我们需要两个计数器,程序首先写成了

\begin{lstlisting}[language=c]
    #include <stdio.h>

    int main() {
        int row;
        int col;
        scanf("%d %d", &row, &col);

        int i = 0;
        int j = 0;
        while (i < row) {
            while (j < col) {
                ++j; 
            }
            ++i;
        }
        
        return 0;
    }
\end{lstlisting}

外层循环控制的是循环几次,即输出几行,内层则控制一行输出多少个1。因此,内层循环的时候要输出1,外层则要在每次循环结束的时候输出一个换行符。所以代码应该写成

\begin{lstlisting}[language=c]
    int i = 0;
    int j = 0;
    while (i < row) {
        while (j < col) {
            printf("1");
            ++j; 
        }
        printf("\n");
        ++i;
    }
\end{lstlisting}

我们试着运行一下,比如输入3 4,发现运行结果是这样的

\begin{mdframed}
\begin{verbatim}
    1111




\end{verbatim}
\end{mdframed}

为什么后面没有1而是只有换行?观察我们的代码,发现在外层第一次循环结束后,j的值等于row,因此在外层进入第二次循环的时候,不会进入内层的循环。为了解决这个问题,我们需要在外面每次循环结束的时候重置j的值,即将代码改成

\begin{lstlisting}[language=c]
    int i = 0;
    int j = 0;
    while (i < row) {
        while (j < col) {
            printf("1");
            ++j; 
        }
        printf("\n");
        j = 0;
        ++i;
    }
\end{lstlisting}

这次再次运行结果就对了。

使用while写太麻烦了,每次都要考虑定义临时变量和变量的自增与重置,我们可以使用for语句重写

\begin{lstlisting}[language=c]
    #include <stdio.h>

    int main() {
        int row;
        int col;
        scanf("%d %d", &row, &col);

        for (int i = 0; i < row; ++i) {
            for (int j = 0; j < col; ++j) {
                printf("1");
            }
            printf("\n");
        }

        return 0;
    }
\end{lstlisting}
