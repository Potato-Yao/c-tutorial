\section{数组}

我们有时需要储存很多变量。如果通过写很多个变量来实现这个功能,不仅难以对每个变量进行储存和读取,也有很强的限制性,因为若想改变变量的数量就需要重新编写代码。因此,c语言提供了一种叫做数组的数据结构用来储存很多变量。

数组就像是一列火车,每一节车厢都装着一个变量。在我们需要获取某个变量的时候,只需要知道它位于第几号车厢即可。一样的,我们想要改变某个变量,只需要知道它的车厢号即可。一列火车有它的长度,我们的数组也有它的长度,并且一个数组只能装一种类型的数据。我们使用数组就可以解决上一段提到的问题了。

数组就是一种特殊的变量。它的声明语法是\textit{数据类型 数组名[数组长度]}。这里的数组长度指的是拥有的元素个数。比如我们要声明一个拥有4个元素的,装载整型变量的数组,就可以这样写:

\begin{lstlisting}[language=C]
    int array[4];
\end{lstlisting}

如何赋值呢?对数组赋值一般叫做数组的初始化。如对于上面声明的数组,可以这样初始化:

\begin{lstlisting}[language=C]
    int array[4] = {1, 2, 3, 4};
\end{lstlisting}

注意,数组的声明和初始化要写在一起。
